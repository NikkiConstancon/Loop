\subsubsection{Notifications Requirements}
        	\begin{enumerate}
            	\item{Location of Notifications Subsystem\\ The Notification functionality will be located on the cloud server as it is more reliable and has easier access to patient data rather than having it on each mobile application where it increases the application size.}
            	\item{Real-time Monitoring of Patient Data\\Patient data received is compared to stats that have been deemed normal before the data is stored in the Cassandra database. Deviations from the norm trigger notification functionality, but data is stored as normal anyway.}
                \item{Judging severity\\Notification functionality will then assess the severity of the patient data that triggered the response and determine its severity.}
                \item{Reacting Accordingly Based On Severity\\ Once severity is established, the Notification functionality will decide on the appropriate response based on the severity. For example, if the patients blood sugar level is slightly bellow average, a low level response will be issued.}
                \item{Types of Responses\\The responses that will be chosen from will range from a small notification on the application itself, to an urgent alert with sound, to an emergency SMS on the caretakers mobile devices.}
                \item{Format of Notifications\\Notifications will be precise, short and as to the point as possible. Especially for emergency alerts. They may also provide minimal advice for the specific situation but should definitely not be considered replacements for medically professional approaches. They are to be read quickly and with ease so as to minimize the time to react appropriately.}
                \item{Notify Nearest Medical Professional\\ If severity is judged to be too great for a caretaker to handle, the application may notify the nearest medical facility with all needed details such as address and condition of patient.}
            \end{enumerate}
