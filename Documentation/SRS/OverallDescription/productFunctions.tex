The main function of the ReVA system will be to monitor a bed ridden patient's vitality and report on that patient's status.
\begin{enumerate}

\item Data Collection: \\
Collect sensory input from peripheral devices. This data will be formatted to the correct medical format. 

\item Recording Data:\\
Put the collected data into a neatly formatted (EMR format) database for easy retrieval and easy calculation. This is to be able to provide the history of the patient's vitality as well as providing an anonymous source for researchers to analyse the data. The calculations are for the statistics that could be things like the average or a graph which tracks the data in a certain time period.

\item Reporting Real-Time Data:\\
Display real-time data from peripheral sensors on the mobile application. The application will be able to give the user as up-to-date data as possible. This means that a device data may be updated as quickly as it changes to provide accurate and real time data anyone who has authority to view the data.

\item Access\\
Control access rights and enforcing authentication before gaining access to collected and real-time patient data. A user must be registered and logged in before being able to see any data whatsoever. An Admin person has the right to add/remove users from the system, and also grant admin rights to certain users.
 
\item Notifications\\
Analyse real-time patient data and report on fluctuations from the norms and suggest possible courses of action. There will be a set of criteria that outline emergencies or things which are worthy of notification. After analysis, the severity of the emergency will be found by means of assigning it a certain code or level. Then the appropriate people, whether a professional doctor or the caretaker or someone else, will be notified about what the stats are showing and about what the problem seems to be.
\end{enumerate}