\documentclass[12pt]{article}
\usepackage{graphicx}
\usepackage{eso-pic}
\usepackage{ragged2e}
\renewcommand\thepage{- \arabic{page} -}
%Logo
\newcommand\Highlight{%
	\put(0,150){%
		\parbox[b][\paperheight]{\paperwidth}{%
		\vfill
		\centering
		\includegraphics[width=\paperwidth, height=10cm]{logo.jpg}%
		\vfill
}}}
%Swirl
\newcommand\Swirl{%
	\put(50,270){%
		\parbox[b][\paperheight]{\paperwidth}{%
		\vfill
		\centering
		\includegraphics[width=20cm, height=20cm]{background.png}%
		\vfill
}}}
\usepackage{graphicx}
\graphicspath{{../images/}}

\AddToShipoutPicture*{\Highlight}
\AddToShipoutPictureBG{%
	\ifnum\value{page}>1
	\AtPageLowerLeft{\Swirl}%
    \fi
    }%
    
\begin{document}

{\fontfamily{phv}\selectfont % change phv to get new fonts for whole document
\font\myfont=cmr12 at 20pt

\begin{center}


\begin{minipage}{0.75\linewidth}


\vspace*{250pt}
\title{ \rule{\linewidth}{2pt} \\
\textbf{\normalfont\fontsize{35}{35}\scshape\selectfont IoT HomeCare System}\\
\textbf{\normalfont\fontsize{35}{35}\scshape\selectfont Testing}\\}
\author{   
        Hristian Vitrychenko\\
        Nikki Constancon \\
        Juan du Preez\\
        Gregory Austin \\
        Marthinus Richter 
}
\date{\today \\ \rule{\linewidth}{2pt}}


\maketitle
\thispagestyle{empty}

\end{minipage}
\end{center}
\pagebreak
	\section{Introduction}
	
	The testing report is designed to describe all tests that have been done on the system and future tests to be done on the system. 
	
		\subsection{Purpose}
		
		The purpose of this document will be to list and describe all of the tests for the ReVA system, how they are carried out, what their purpose is and what use case they are related to. 
	
		\subsection{Definitions, Acronyms, and Abbreviations}
			
			
			\subsubsection{Acronyms}
			
			\begin{itemize}
				
				\item \textbf{UI} \textbf{\textit{(User Interface)}} \\
				\newline
				The means by which the user and a computer system interact, in particular, the use of input devices and software.
			
			\end{itemize}
		
			\subsubsection{Definitions}
		
			\begin{itemize}
				
				\item \textbf{Unit Test}\\
				\newline
				Unit testing is a software development process in which the smallest testable parts of an application, called units, are individually and independently scrutinised for proper operation.\\
				
			\end{itemize}
		
		\subsection{Additional Information}
		
		The code being tested is written by different developers, thus this document serves as a way in order to review the functionality provided, and to provide information about future tests on current functionality.\\\\
		

	\pagebreak
	\section{Functional requirements}
        
	\subsection UC1: A user can view real time data gathered from a patient.

	\subsection UC2: A user can view historical data detailing a period of time.
	
	\subsection UC3: A user can view statistical data from over a period of time.

	
	\subsection UC4: A user can register to create an account and link to a particular patient.
	
	\subsection UC5: A user can log in to his or her account.
	
	\subsection UC6: A user can log out of his or her account.
	
	\subsection UC7: A user can manage his or her account information.		

	\subsection UC8: An admin user can manage other users' access rights.


	\subsection UC9: A user, caregiver, or doctor can get notifications if something is wrong according to the data.

	
	\subsection UC10: A user can see all previous notifications.

	\subsection UC11: A user can view general advice on the App.
	
	\subsection UC12: A user can add or remove medical persons' contact details to the app.

	\subsection UC13: A user can configure devices remotely.

		
	\subsection UC14: A researcher can view the data in the database anonymously.	


    \section{Non-functional requirements}
    	\subsection{Usability test}
    	\subsection{Accessibility test}

    \section{Platform compatibility test}
   
			
\end{document}