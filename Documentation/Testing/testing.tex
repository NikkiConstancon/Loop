\documentclass[12pt]{article}
\usepackage{graphicx}
\usepackage{eso-pic}
\usepackage{ragged2e}
\renewcommand\thepage{- \arabic{page} -}
%Logo
\newcommand\Highlight{%
	\put(0,150){%
		\parbox[b][\paperheight]{\paperwidth}{%
		\vfill
		\centering
		\includegraphics[width=\paperwidth, height=10cm]{logo.jpg}%
		\vfill
}}}
%Swirl
\newcommand\Swirl{%
	\put(50,270){%
		\parbox[b][\paperheight]{\paperwidth}{%
		\vfill
		\centering
		\includegraphics[width=20cm, height=20cm]{background.png}%
		\vfill
}}}
\usepackage{graphicx}
\graphicspath{{../images/}}

\AddToShipoutPicture*{\Highlight}
\AddToShipoutPictureBG{%
	\ifnum\value{page}>1
	\AtPageLowerLeft{\Swirl}%
    \fi
    }%

\begin{document}

{\fontfamily{phv}\selectfont % change phv to get new fonts for whole document
\font\myfont=cmr12 at 20pt

\begin{center}


\begin{minipage}{0.75\linewidth}


\vspace*{250pt}
\title{ \rule{\linewidth}{2pt} \\
\textbf{\normalfont\fontsize{35}{35}\scshape\selectfont IoT HomeCare System}\\
\textbf{\normalfont\fontsize{35}{35}\scshape\selectfont Testing}\\}
\author{
        Hristian Vitrychenko\\
        Nikki Constancon \\
        Juan du Preez\\
        Gregory Austin \\
        Marthinus Richter
}
\date{\today \\ \rule{\linewidth}{2pt}}


\maketitle
\thispagestyle{empty}

\end{minipage}
\end{center}
\pagebreak
	\section{Introduction}

	The testing report is designed to describe all tests that have been done on the system and future tests to be done on the system.

		\subsection{Purpose}

		The purpose of this document will be to list and describe all of the tests for the ReVA system, how they are carried out, what their purpose is and what module and related artifact(s) they are related to.

		\subsection{Structure of the document}

		Each subsystem of ReVA will be addressed individually with the specific tests for each module concerned with that subsystem. The subsystems are:
		\begin{itemize}
			\item Real-Time Subsystem \\ This includes the modules: Data Collection,  Pub/Sub Server, Data Pull, and User Interface
			\item Data Storage Subsystem \\ This includes the modules: Data Storage, Pub/Sub Server
			\item History/Statistics Subsystem \\ This includes the modules: Data Storage, Statistics, Pub/Sub Server, and User Interface
			\item User Management Subsystem \\ This includes the modules: User Management, User Interface
			\item Notification Subsystem \\ This includes the modules: Pub/Sub Server, Notification, and User Interface
			\item Advice Subsystem \\ This includes the modules: Advice and User Interface
		\end{itemize}
		\subsection{Definitions, Acronyms, and Abbreviations}


			\subsubsection{Acronyms}

			\begin{itemize}

				\item \textbf{UI} \textbf{\textit{(User Interface)}} \\
				\newline
				The means by which the user and a computer system interact, in particular, the use of input devices and software.

			\end{itemize}

			\subsubsection{Definitions}

			\begin{itemize}

				\item \textbf{Unit Test}\\
				\newline
				Unit testing is a software development process in which the smallest testable parts of an application, called units, are individually and independently scrutinised for proper operation.\\

			\end{itemize}

		\subsection{Additional Information}

		The code being tested is written by different developers, thus this document serves as a way in order to review the functionality and tests done, and to provide information about future tests on future or current functionality.\\\\


	\pagebreak

	\section{Real-time subsystem}
	\textbf{Implementation status:} implemented \\
	\textbf{Primary Actors:} All users \\
	\textbf{Functionality:} View real time data on patient(s)\\
	\subsubsection{Description of current tests}
	The data streaming consists of Raspberry pi's streaming to servers, and servers streaming that data out and the application displaying those data streams. We do this using the Nodejs package Zetta (which is already been tested) for the Data Collection and the Pub/Sub Server, and an application that someone created and tested to recieve zetta streams. These have all been tested individually already, thus to create our own unit tests would be redundant.
	\subsubsection{Future tests:}
	Future tests will include integration testing, functionality testing (functionality vs requirements), performance testing (seeing capacity and application performance) and some alpha testing (testing it ourselves to potentially find problems).
	\subsubsection{Data Collection}
	\textbf{Comment:} At the moment the Pi collects data from mock devices. This data generated and streamed using the zetta architecture and therefore has already been tested with Unit tests thoroughly. (Zetta is an open source Nodejs project purposed for IoT)
	\subsubsection{Pub/Sub Server}
	\textbf{Comment:} The Pub/Sub server uses zetta architecture as well and that's how it communicates with the pi (recieves streams being published) and any requesting devices looking to subscribe. Zetta already is established and has tests.
	\subsubsection{Data Pull}
	\textbf{Comment:} The way the data is pulled is by using the architecture of the open source Zetta-Starter-Android-Application developed to interface with the Zetta API that's generated by the server. This has also been tested, so testing is redundant.
	\subsubsection{User Interface}
	\textbf{Comment:} The user interface only displays data, and therefore it is easy to verify what is being displayed and that it is working as expected.

	\section{Data Storage Subsystem}
	\textbf{Implementation status:} partly-implemented
	\subsubsection{Data Storage, interfaced with the Pub/Sub server}
	\textbf{Implementation status:} partly-implemented \\
	\textbf{Primary Actors:} users, admin users \\
	\textbf{Functionality} Create, Remove, Update or Delete items related to data storage, e.g., timestamped human vital data. Current functionality is that the Data Storage module can recieve streams from the Pub/Sub server.  \\
	\subsubsection{Description of current tests}
	\textbf{Precondition:} The Pub/Sub server module is tested to ensure that arguments passed to the database manager are correctly formatted, i.e., that the json conforms to the database model. \\

	\textbf{Postcondition:} Ensure that intended parameters are added to the database after successful query. \\

	\textbf{Invariant:} Ensure that only intended parameters change within the related managers and database.



	\section{History/Statistics subsystem}
	\textbf{Implementation status:} unimplemented

	\section{User Management Subsystem }
	\textbf{Implementation status:} partly-implemented
	\subsubsection{User Management}
	\textbf{Implementation status:} partly-implemented \\
	\textbf{Primary Actors:} users, admin users \\
	\textbf{Functionality} Create, Remove, Update or Delete items related to data storage, e.g., user info, users, user relationships etc.  \\
	\subsubsection{Description of current tests}
	\textbf{Precondition:} The user manager is tested to ensure that arguments passed to the database manager are correctly formatted, i.e., that the json conforms to the database model. \\
	\textbf{Postcondition:} Ensure that intended parameters are added to the database after successful query. \\
	\textbf{Invariant:} Ensure that only intended parameters change within the related managers and database.
	\subsubsection{User Interface}
	\textbf{Implementation status:} partly-implemented \\
	\textbf{Primary Actors:} non-users \\
	\textbf{Functionality} Type in details and register to become a new user. There is only local validation and forms currently implemented. Client-server communication for registration and user management is not yet implemented.
	\subsubsection{Description of current tests}
	There are unit tests for each input that test the validation of those inputs. So there are tests for email, passwords, usernames etc. All to make sure that the validation is correct.
	\subsubsection{Future tests:}
	Unit tests that verify client-server communication is happening correctly.

	\section{Notification Subsystem }
	\textbf{Implementation status:} unimplemented

	\section{Advice Subsystem}
	\textbf{Implementation status:} unimplemented

	\section{Non-functional requirements}
		\subsection{Usability test}
		\subsection{Accessibility test}

	\section{Platform compatibility test}



\end{document}
